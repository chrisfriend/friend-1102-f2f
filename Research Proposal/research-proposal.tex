\documentclass[11pt]{amsart}	%defines this as an article
\usepackage{chrisfriend-comp} %provides formatting declarations for page, headers, figures, textcolor, comments, and bibliographic styles
\usepackage{chrisfriend-OTF-support} %provides support for OTF system fonts; incompatible with latex, rtf2latex, & ht4latex
%\usepackage[utf8]{inputenc} %support for smallamp?

%\usepackage{draftwatermark}

%\usepackage{tabularx}
\usepackage{tabulary} % allows for the tables I make rubrics with
%\usepackage{supertabular}
\usepackage{xtab} % allows tables to span pages
\usepackage{booktabs} % allows fancy lines in tables
\usepackage{rotating} % allows landscape tables
\usepackage{lscape} % allows rotated longtables
\usepackage{multirow} % allows rowspanning
\usepackage{enumitem} % helps with the overview
%\usepackage{paralist}

\title[Research Proposal]{Assignment Sheet: Research Proposal}
\chead{\scriptsize{\MakeUppercase{Research Proposal}}}

\begin{document}
%\bibliographystyle{abbrv}
\thispagestyle{empty}

\vspace{-2in}
\begin{center}
\huge
\includegraphics[height=1.5\baselineskip]{pegasus.pdf}

\textbf{Assignment Sheet: Research Proposal}

{\normalsize Chris Friend • \textsc{enc1102} • Fall 2013}
\end{center}
\vspace{1.5\baselineskip}

\section{Background \& Purpose} % (fold)
\label{sec:background}
The last paper you submitted was brainstorming. Now it's time to let the dust settle and show that you've focused in on some realistic plans for conducting your study. After the readings and discussions in class, plus your initial explorations, you should have a good sense of where your project might take you. This document is your chance to share that sense with your workgroup.

The major objective of this assignment, then, is to show that you have identified a clear research question or problem and have settled on a plan of action for exploring it. You should strike a more confident tone here, as though you are convincing your instructor to permit your study to continue. In fact, that is the essence of a proposal. \textbf{Make your study important in the eyes of your audience—your workgroup and instructor.}

Write a proposal (probably around 1,000 words) that describes your question or problem, explains what you hope to learn by investigating it, and describes the people or groups who appear to be interested in it. Show that you have thought out your problem, seen potential significance of it, and seen potential application to other people. Also, locate a sample of sources you could explore to find the existing discussion about your issue. Your list should be in the form of an \textsc{apa}- or \textsc{mla}-formatted list of preliminary sources you plan to explore---practice your citation format.
% section background (end)

\section{Procedure} % (fold)
\label{sec:procedure}
This document needs to show more polish than the previous one, but more importantly, it needs to show evidence of more deliberate thinking. Your job is to show that you have done these things:
\begin{enumerate}
	\item Consider the question or problem you had identified in your Brainstorming Audit. Refine or clarify it, based on the continued thinking you've done\footnote{Use the process illustrated by the \href{http://library.wlu.ca/tutorials/research_question}{Developing a Research Question} presentation from the Laurier Library at U of Waterloo.}. Explain what makes this question or problem worth further investigation. (Anticipate the ``So what?'' question.) Briefly consider why this question or problem has not been resolved.
	\item Consider what you hope to learn by investigating this question or problem. What value do you hope to provide to others who may be interested? 
	\item Conduct some very initial exploration of your problem or question. Determine who else is talking about this problem or issue and who else cares about this issue besides just you. How do you know they are interested? What evidence have you seen?
	\item Conduct an initial search for sources that you might use in your investigation. Consider both scholarly and popular, with the balanced determined by the nature of your question/problem. You don’t have to read these now, but provide a list of what you plan to read.
\end{enumerate}
% section procedure (end)

\section{Format} % (fold)
\label{sec:format}
Research proposals commonly precede actual research work in academic inquiry, making them familiar to academics and other researchers. As a result, you have access to helpful resources that offer guidance and examples to help you build your document. Refer to pages 297--307 in your Greene/Lidinsky text for specific details and an sample document. Consider using section headings to make your proposal clearer and easier to digest. Feel free to print your research question in bold text to ensure I cannot miss it when reading your proposal.
% section format (end)

\section{Evaluation} % (fold)
\label{sec:evaluation}
Your Brainstorming Audit was a rough outline of your initial thinking on the project. This assignment adds formality and a clearer sense of process and intended outcome. Your thinking for this project is getting more serious; be sure that's reflected in the presentation of your paper. Show that you are thoroughly processing the material from class and the investigations you're doing. Specific evaluation guidelines are shown in Table~\ref{tab:rubric}.
% section evaluation (end)

\begin{table}[b]
	\caption{Evaluation of Research Proposal}\label{tab:rubric}
	\small
\begin{tabulary}{\textwidth}{rLLLL}
	\toprule  & \textbf{\textsc{Question/Problem Justification}} & \textbf{\textsc{Inquiry Progress}} & \textbf{\textsc{Outcome Consideration}} & \textbf{\textsc{Potential Sources}}\\
\midrule	\textbf{Excellent} & Provides relevance to the field being studied and shows clear relation to a concerned audience & Documents the discovery process that led to current thinking & Clearly explains expected outcomes and consequences of research & Identifies a variety of relevant sources that will potentially help investigations \\
\midrule	\textbf{Adequate} & Provides a reason for the research; lacks relation to outside parties & Includes steps taken but lacks sense of discovery as process & Identifies expected outcomes; does not extend into foreseen consequences & Lists sources that are marginally relevant or limited in scope \\
\midrule	\textbf{Poor} & Research topic is presented as naturally important or without the need to support & Research process consists of in-class activities or little initiative & Outcomes of the study are absent, unclear, or impractical & Sources are absent, irrelevant, or inappropriate to the question/problem \\
	\bottomrule
\end{tabulary}
\end{table}
% section rubric (end)

\end{document}
