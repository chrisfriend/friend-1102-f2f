%% For help on available commands, see the following documentation files, available at CTAN (http://www.ctan.org/).
%%    usepackage{acronym, babel, biblatex, biblatex-apa, biblatex-apa-test, booktabs, calc, color, comment, csquotes, datetime, docmute, fontspec, footmisc, geometry, graphicx, hyperref, ifxetex, inputenc, isodateo, lineno, longtable, memman, mparhack, multirow, nameref, paralist, siunitx, suffix, svn-multi, todonotes, ulem, xcolor, xltxtra, xtab, xunicode}
%%
%% Available options to the eyerdoc-common style include:
%%	apa		uses APA6-like citations and references (one of apa or mla must be specified)
%%	mla		uses MLA6-like citations and references elements (one of apa or mla must be specified)
%%	nosvn	operates without subversion versian control features
%%  strict	uses strict apa or mla geometry and design elements
%% Additionally, the eyerdoc-common style requires the memoir class and uses its article, draft, and final options
%% For more information about the eyerdoc styles, see http://www.eyer.us/eyerdoc
%%

% Preamble and document properties (fold)
\documentclass[10pt,article,oneside]{memoir}
\usepackage[apa,nosvn]{eyerdoc-common} % for information on eyerdoc packages, see http://www.eyer.us/eyerdoc
\usepackage{eyerdoc-chrisfriend-fonts}
\bibliography{bibliography}
\input{acronyms}
\usepackage{fullpage}

\author{Friend's 1102 Classes}
\briefauthor{Friend}
\institution{Spring 2013}
\title{Research Paper Review}% use \par to break long titles
\runninghead{Research Paper Review}% this is used in running headers and in pdf metadata
%\date{\printdate{2013-04-17}}
% Preamble and document properties (end)

\begin{document} % start of document (fold)
% frontmatter (fold)

\settitle % sets the document title and other relevant information

%\begin{abstract}
%	Place abstract text here...
%\end{abstract}

\settitle % sets the title after the title page when applicable
%\setlistoftodos% sets list of \todos; automatically disabled when memoir class is called with final option
%\settableofcontents% sets table of contents
%\setlistoftables% sets list of tables
%\setlistoffigures% sets list of figures

% frontmatter (end)

Check for the following in each research paper. The section titles below aren't required, but they are traditional for this genre. Regardless of section headings, you should ``feel'' the transition from one purpose to another as you read through the paper.

\begin{multicols}{2}
\begin{compactitem}
	\item \textbf{Title}
	\begin{compactitem}
		\item Does the title indicate both the topic and the aspect being discussed?
		\item Would the title benefit from a ``medial colon'', starting with a very brief or catchy phrase and continuing with a statement of what's being explored in the research?\footnote{Examples from \href{https://twitter.com/academictitles}{@academictitles} on Twitter. Yes, that actually exists. And yes, it's rather funny.}
		\begin{compactitem}
			\item ``Hyde and Sikh: Literary Dualism in Sub-Continental Religions''
			\item ``(You Were) Right About Mao, Funk Soul Brother: Predicting the Interpretation of Zedong's Legacy Through 1970's Blacksploitation Films''
			\item ``Fluoxeteenage Mutant Ninja Turtles: Success Rates of Generic Anti-Depressants for Amphibious Crime Fighters''
			\item ``Feminism Ain't Funny: Woman as `Fun-Killer,' Mother as Monster in the American Sitcom''
			\item ``iLess in Gaza: Apple, Hegemony and the loss of corporate control in Palestine''
			\item ``Seder? I Hardly Knew Her! Examining the Historical Roots of the Comic Pun in the 1st Book of the Tanakh''
			\item ``It's Not a Table for One---I'm Leaving it for Elijah: The Passover Guide to Dining Out Alone''
		\end{compactitem}
	\end{compactitem}
	\item \textbf{Introduction}
	\begin{compactitem}
		\item What do other researchers currently think about the topic of this paper?
		\item How does the author's new research fit in with the existing work?
		\begin{compactitem}
			\item extends what they started
			\item challenges what they concluded
			\item answers what they asked for
			\item confirms/echoes what they found
		\end{compactitem}
		\item Do you know what the author hopes to do with the paper? (In other words, where is the thesis?)
	\end{compactitem}\columnbreak
	\item \textbf{Methodology}
	\begin{compactitem}
		\item Does the author explain how the new research was conducted?
		\item Do you know why the author chose that approach? Is it a valid reason?
		\item Do you know what kinds of people were chosen to participate?
		\item Do you know why those people were chosen? Is it a valid choice?
	\end{compactitem}
	\item \textbf{Results}
	\begin{compactitem}
		\item Does the author show evidence of results through quotes, statistics, etc.?
		\item Are the results presented in a useful and readable manner, or are they jumbled or monotonous?
		\item Do the results align with the author's goal from the Intro §?
		\item Does the author help you make sense of the results, or is it up to you?
		\item Does the author highlight the important, surprising, or expected results, or are they all treated equally?
	\end{compactitem}
	\item \textbf{Discussion}
	\begin{compactitem}
		\item Does the author explain why the findings from the Results § are important?
		\item What changes, actions, or next steps does the researcher suggest?
		\begin{compactitem}
			\item Fix a problem by…
			\item Adjust a plan so that…
			\item Stop/start doing such-and-such.
			\item Think of such-and-such in a different way.
			\item Ask so-and-so to start/stop…
		\end{compactitem}
		\item Are you left feeling convinced?
		\item Does the end address the concerns of the Intro §?
		\item Is the initial research question addressed? (It needn't be \emph{answered}, but it must be \emph{addressed}.)
	\end{compactitem}
\end{compactitem}
\end{multicols}

% backmatter (fold)

%\setappendix*% begins alphabetic section numbering and prints "Appendi(x|ces)"; starred command prints nothing
%\setreferences % typesets the reference list; if no references are used, the list will not be typeset

% backmatter (end)

\end{document} % (end)