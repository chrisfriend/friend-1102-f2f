\documentclass[10pt,oneside]{amsart}	%defines this as an article
\usepackage{chrisfriend-comp} %provides formatting declarations for page, headers, figures, textcolor, comments, and bibliographic styles
\usepackage{chrisfriend-OTF-support} %provides support for OTF system fonts; incompatible with latex, rtf2latex, & ht4latex
%\usepackage[utf8]{inputenc} %support for smallamp?

%\usepackage{draftwatermark}

%\usepackage{tabularx}
\usepackage{tabulary} % allows for the tables I make rubrics with
%\usepackage{supertabular}
\usepackage{xtab} % allows tables to span pages
\usepackage{booktabs} % allows fancy lines in tables
\usepackage{rotating} % allows landscape tables
\usepackage{lscape} % allows rotated longtables
\usepackage{multirow} % allows rowspanning
\usepackage{enumitem} % helps with the overview
%\usepackage{paralist}

\title[Academic Research Report]{Assignment Sheet: Academic Research Report}
\chead{\scriptsize{\MakeUppercase{Academic Research Report}}}

\begin{document}
%\bibliographystyle{abbrv}
\thispagestyle{empty}

\vspace{-2in}
\begin{center}
\huge
\includegraphics[height=1.5\baselineskip]{pegasus.pdf}

\textbf{Assignment Sheet: Academic Research Report}

{\normalsize Chris Friend • \textsc{enc1102} • Fall 2013}
\end{center}
\vspace{1.5\baselineskip}

\section{Background and Purpose} % (fold)
\label{sec:background}
Now that you have gathered existing information and created new knowledge about your chosen issue, it's time to begin stating your case with those ideas. Your job is to report on your findings in a traditional academic research paper, styled after the journal articles you gathered for your secondary research. This paper is designed to show that you can express your conclusions and findings using a well-structured and clearly articulated academic argument. Think of this as your opportunity to contribute to the academic conversation you highlighted in your framing synthesis. The authors you discussed there are your audience for this paper. You've studied the academic conversation already; now you will participate in it.
% section background (end)

\section{Procedure} % (fold)
\label{sec:procedure}
\begin{enumerate}
	\item Revisit your previous assignments, particularly your Research Proposal and Secondary Research Report. What had you initially set out to discover? What were your research questions? Determine how you currently answer those questions. Did you make the discovery you hoped for?
	\item Find your reason for writing. What would you like to say to the academics studying this field?
	\item Using all the resources you have developed this semester, talk back to those authors. Create a document that argues your point \emph{using the language, methods, and forms common to the academic audience you first explored}. This means mimicking the structure, tone, and style of the documents you studied earlier.
	\item Make deliberate use of ethos, logos, and pathos when you construct your argument.
	\item Let your readers know about the benefits or implications of your work.
\end{enumerate}
% section procedure (end)



\begin{table}[b]
	\caption{Evaluation of Academic Research Report}\label{tab:rubric}
\begin{tabulary}{\textwidth}{rLLL}
	\toprule  & \textbf{\textsc{Argument}} 
%	& \textbf{\textsc{Genre Sets}} 
	& \textbf{\textsc{Integration}}
	& \textbf{\textsc{Audience}} \\
\midrule	\textbf{Excellent} 
& Presents convincing argument, deftly \textbf{balancing} ethos, logos, \& pathos
& \textbf{Balances} outside sources \& primary research, presenting each when appropriate
& Meets expectations of academic audience (tone, citation, etc.) w/ \textbf{rhetorically appropriate} delivery  \\
\midrule	\textbf{Adequate} 
& Presents argument with deliberate attempt to \textbf{build} ethos, logos, \& pathos 
& Shows how primary research \textbf{follows from} findings of secondary research sources
& Demonstrates \textbf{attention} to academic expectations for delivery; \textbf{attempts} sophistication \& maturity \\
\midrule	\textbf{Poor} 
& Insufficient logos/support to carry unfounded claims; \textbf{absence} of ethos \& pathos
& \textbf{No connection} made between primary \& secondary research
& Text too personal, casual, or non-issue-focused; \textbf{inappropriate} for academic situation \\
	\bottomrule
\end{tabulary}
\end{table}
% section stake-eval (end)


\end{document}




\section{Evaluation} % (fold)
\label{sec:eval}
This section to be determined via class discussion.
% section stake-eval (end)


\section{Formatting} % (fold)
\label{sec:formatting}
Because your goal is to fit your writing into the existing academic conversation, formatting is actually rather important. Use a specific academic style guide. An \textsc{mla}-formatted template is available from Webcourses, if that format meets your needs; use a style appropriate for the field you are writing to. Unless you can justify a different approach, your work should include:
\begin{itemize}
	\item double-spaced lines;
	\item one-inch margins on all sides and half-inch indents for paragraphs;
	\item a 12-point typeface with serifs (like Times New Roman, \emph{not} Calibri);
	\item parenthetical citations and a Works Cited or References page, as appropriate; and
	\item appendices for data collected, including any personal interviews or email correspondence you used to learn about the genre sets.
\end{itemize}
% section formatting (end)
