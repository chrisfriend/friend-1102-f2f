\documentclass[10pt,oneside]{amsart}	%defines this as an article
\usepackage{chrisfriend-comp} %provides formatting declarations for page, headers, figures, textcolor, comments, and bibliographic styles
\usepackage{chrisfriend-OTF-support} %provides support for OTF system fonts; incompatible with latex, rtf2latex, & ht4latex
%\usepackage[utf8]{inputenc} %support for smallamp?

\usepackage{draftwatermark}

%\usepackage{tabularx}
\usepackage{tabulary} % allows for the tables I make rubrics with
%\usepackage{supertabular}
\usepackage{xtab} % allows tables to span pages
\usepackage{booktabs} % allows fancy lines in tables
\usepackage{rotating} % allows landscape tables
\usepackage{lscape} % allows rotated longtables
\usepackage{multirow} % allows rowspanning
\usepackage{enumitem} % helps with the overview
%\usepackage{paralist}

\title[Primary Research Report]{Assignment Sheet: Primary Research Report}
\chead{\scriptsize{\MakeUppercase{Primary Research Report}}}

\begin{document}
%\bibliographystyle{abbrv}
\thispagestyle{empty}

\vspace{-2in}
\begin{center}
\huge
\includegraphics[height=1.5\baselineskip]{pegasus.pdf}

\textbf{Assignment Sheet: Primary Research Report}

{\normalsize Chris Friend • \textsc{enc1102} • Fall 2013}
\end{center}
\vspace{1.5\baselineskip}

\section{Background and Purpose} % (fold)
\label{sec:background}
In previous assignments, you have identified the conversation taking place around your research topic, and you have identified what you want to accomplish through this research project. In this assignment, you will create new knowledge that you can then contribute to the existing conversation you reviewed. Along the way, you will improve your rhetorical-analysis skills and determine potential avenues of distributing what you learn to interested groups.
% section background (end)

\section{Procedure} % (fold)
\label{sec:procedure}
For this assignment, the hard work happens before you start writing---creating this report is simply documenting the work you did before-hand. Refer to the \href{https://webcourses.ucf.edu/courses/982699/wiki/primary-research-module-4-intro}{Module Introduction} page in Webcourses for specific details on how to prepare for this assignment. The activities in this Module encourage the thinking and produce the data you will use to compose this analysis. You may wish to divide your document into Stakeholder, Genre, and Data Analysis sections, accordingly, but the overall goal is to analyze the situation in which your primary research exists. Present your answer to these questions: \textbf{What new knowledge have you created? Who will listen to you? How could you get that information to those people?}

\begin{comment}
	In brief, you need to accomplish these tasks before you can write the report:
	\begin{enumerate}
		\item Brainstorm some of the people who might be the most interested in your research question or problem. Choose one or two of these groups of stakeholders to focus on as a possible audience for your final project. (In short, who cares about your opinions? Whom do you need to persuade?)
		\item Determine the “genre sets” used by that group of stakeholders. (How do those people communicate? What persuades them?)
		\item Analyze the characteristics of one genre used by your stakeholder. Determine what you need to know to be able to competently write in it for your final project. (Create a genre memo in a group of your peers who chose the same genre.)
		\item Conduct original research to add to the current knowledge on the topic.
		\item Analyze the data you collect and present it meaningfully to an audience with little familiarity with your topic (i.e. your peers).
	\end{enumerate}
\end{comment}

\subsection{Stakeholder Analysis} % (fold)
\label{sub:stakeholder_analysis}
Write an analysis (likely around 500 words) in which you outline all the stakeholders you have considered for your research interests. Then, conduct a more focused analysis of one or two of those groups of stakeholders that you believe would best benefit from your research findings. Consider questions such as these:
	\begin{itemize}
		\item What makes each individual or group a stakeholder in this topic?
		\item  What is their particular interest in the problem or question you are researching?
		\item  What can they do in response to your findings/proposals/ideas?
	\end{itemize}
% subsection stakeholder_analysis (end)

\subsection{Genre Analysis} % (fold)
\label{sub:genre_analysis}
Consider the environment of the stakeholders you identified in your \nameref{sub:stakeholder_analysis}. Analyze how writing works in that environment to help exchange or receive information. To avoid guessing, you likely need to conduct interviews %\footnote{See Greene \& Lidinsky pp. 307--310 for guidance on conducting effective interviews.}
and collect samples of unfamiliar texts from members of your stakeholder group. This is an opportunity to ``get your hands dirty'' for your research. Then, write a two- to three-page analysis of their methods of written communication.
	\begin{enumerate} % For concise print version, change to "comment". For extended online version, use "enumerate".
		\item  Identify the ``genre sets'' used by the primary group of stakeholders you focused on in your analysis. Consider questions such as these:
		\begin{itemize}
			\item What texts do these stakeholders routinely read and write? Where do they get their information? 
			\item  What information do they find convincing when making a decision?
		\end{itemize}
		\item Determine which of those genres would be the most suitable for your use when communicating with the stakeholders you chose. Explain why it is the best option. Then, consider questions such as these:\footnote{Taken from \emph{The Norton Field Guide}, pages 10–11, quoted in \citetitle{bawarshi2010genre} by \citeauthor{bawarshi2010genre}, pages 195–96.}
		\begin{enumerate}
	\item \textbf{What is your genre, and does it affect what content you can or should include?}\\ Objective information? Researched source material? Your own opinions? Personal experience?
\item \textbf{Does your genre call for any specific strategies?}\\ Profiles, for example, usually include some narration; lab reports often explain a process.
\item \textbf{Does your genre require a certain organization?}\\ Most proposals, for instance, first identify a problem and then offer a solution. Some genres leave room for choice. Business letters delivering good news might be organized differently than those making sales pitches.
\item \textbf{Does your genre affect your tone?}\\ An abstract of a scholarly paper calls for a different tone than a memoir. Should your words sound serious and scholarly? brisk and to the point? objective? opinionated? Sometimes your genre affects the way you communicate your stance.
\item \textbf{Does the genre require formal (or informal) language?}\\ A letter to the mother of a friend asking for a summer job in her bookstore calls for more formal language than does an email to the friend thanking him for the lead.
\item \textbf{Do you have a choice of medium?}\\ Some genres call for print; others for an electronic medium. Sometimes you have a choice: a résumé, for instance, can be mailed (in which case it must be printed), or it may be emailed. Some teachers want reports turned in on paper; others prefer that they be emailed or posted to a class Web site. If you’re not sure what medium you can use, can you ask your stakeholders?
\item \textbf{Does your genre have any design requirements?}\\ Some genres call for paragraphs; others require lists. Some require certain kinds of typefaces—you wouldn’t use Impact for a personal narrative, nor would you likely use Dr.\ Seuss for an invitation to Grandma’s sixty-fifth birthday party. Different genres call for different design elements.
		\end{enumerate}
	\end{enumerate} % For concise print version, change to "comment". For extended online version, use "enumerate".

%After completing this thorough analysis of both your stakeholders and the genres that they use, you will have determined the audience for your final project, identified how best to reach that audience, and learned how to write appropriately for that discourse community in the context of your rhetorical situation. The more insight you gain from this analysis, the more confidence and clarity you will have going into your final project.
% subsection genre_analysis (end)
% section procedure (end)

\subsection{Data Analysis} % (fold)
\label{sub:data_analysis}
Finally, analyze what you have learned from a data collection of your own. Rather than relying and reporting on the work of others here, you will provide the results of your own research. Because the scenario, methodology, and objective of your project will be distinctive, this assignment sheet cannot provide specific guidance. However, the general process will include:
\begin{enumerate}
	\item Determine what you want to learn. This is best phrased as a non-binary question.
	\item Determine how you could go about finding the answer to that question.
	\item Collect the data using appropriate methods (like interviews, surveys, or observations).
	\item Make sense of that data by analyzing it. What do your results mean? Is your initial question sufficiently answered?
\end{enumerate}

Then, in your written analysis, report on your findings. Explain what you did, what you learned, and what the implications are. This is your chance to present research findings like you read in the articles from your Secondary Research Report. What can you contribute to the conversation?

% subsection data_analysis (end)


\section{Evaluation} % (fold)
\label{sec:eval}
Demonstrate your ability to \emph{analyze} in this document. Show that you have thoroughly examined the potential stakeholders and genres related to your research interests and the findings of your own data collection. Craft a detailed textual analysis of the genre you plan to produce for your final project, and demonstrate that it is relevant to the context and stakeholders you have identified. Be sure to support your analysis with well-organized evidence, documented appropriately. See Table~\ref{tab:rubric} for detailed evaluation criteria.
% section stake-eval (end)

\begin{table}[b]
	\caption{Evaluation of Primary Research Report}\label{tab:rubric}
\begin{tabulary}{\textwidth}{rLLL}
	\toprule  & \textbf{\textsc{Stakeholders}} 
	& \textbf{\textsc{Genres}}
	& \textbf{\textsc{Collected Data}}
	\\
\midrule	\textbf{Excellent} 
& Confidently navigates field of stakeholders; selects based on definitive evidence; clearly relates to research 
& Thoroughly evaluates potential implications \& effectiveness of genre in context of research
& Provides analysis of collected data that contributes to research and convincingly situates results
\\
\midrule	\textbf{Adequate} 
& Considers multiple stakeholders; provides reason for selection; explores relation to research 
& Explores how the genre could work for purposes of research; implications weak or missing
& Reports results of appropriate and legitimate research; implications may be weak or unclear
\\
\midrule	\textbf{Poor} 
& Omits or oversimplifies discussion of options; fails to place in context of research 
& Identifies potential genre but does not justify or analyze
& Questionable data collection; may not address original question
\\
	\bottomrule
\end{tabulary}
\end{table}
% section stake-eval (end)

% section stakeholder_analysis (end)
\end{document}

\section{Formatting} % (fold)
\label{sec:formatting}
Formatting is less critical in this assignment then it was in your Context Analysis; however, be sure to present your report formally, and consider following traditional academic style guidelines. An \textsc{mla}-formatted template is available from Webcourses, if that format meets your needs. Unless you can justify a different approach, your work should include:
\begin{itemize}
	\item double-spaced lines,
	\item one-inch margins on all sides and half-inch indents for paragraphs,
	\item a 12-point typeface with serifs (like Times New Roman, \emph{not} Calibri), and
	\item parenthetical citations and a Works Cited or References page, as appropriate. Be sure to include any personal interviews or email correspondence you used to learn about the genre sets.
\end{itemize}
% section formatting (end)

