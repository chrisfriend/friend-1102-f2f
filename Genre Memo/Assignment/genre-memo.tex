\documentclass[10pt, oneside, twocolumn]{amsart}	%defines this as an article
\usepackage{chrisfriend-comp} %provides formatting declarations for page, headers, figures, textcolor, comments, and bibliographic styles
\usepackage{chrisfriend-OTF-support} %provides support for OTF system fonts; incompatible with latex, rtf2latex, & ht4latex
%\usepackage[utf8]{inputenc} %support for smallamp?

%\usepackage{tabularx}
\usepackage{tabulary} % allows for the tables I make rubrics with
%\usepackage{supertabular}
\usepackage{xtab} % allows tables to span pages
\usepackage{booktabs} % allows fancy lines in tables
\usepackage{rotating} % allows landscape tables
\usepackage{lscape} % allows rotated longtables
\usepackage{multirow} % allows rowspanning
\usepackage{enumitem} % helps with the overview
\usepackage{acronym}
\input{acronyms}
%\usepackage{paralist}
%\usepackage{draftwatermark}

\title[The ``Genre Memo'']{Assignment Sheet: The ``Genre Memo''}
\chead{\scriptsize{\MakeUppercase{The ``Genre Memo''}}}
\lhead{\scriptsize{\textsc{enc 1101}}}

\begin{document}
%\bibliographystyle{abbrv}
\thispagestyle{empty}
%\vspace{-3in}

\twocolumn[
\begin{center}
\huge
{\includegraphics[height=1.5\baselineskip]{pegasus.pdf}}

\textbf{Assignment Sheet: The ``Genre Memo''}

{\normalsize Chris Friend • \textsc{enc1102} • Fall 2013}
\end{center}
\vspace{\baselineskip}
] %Use for column-spanning the title


\section{Background} % (fold)
\label{sec:background}
Throughout the semester, we will work to build your understanding of both the research process and the role of genre in writing situations. These two understandings go hand-in-hand, especially if you consider the varied roles of writing throughout the research process. As you move through this process this term, you will be asked to evaluate the genres we use in terms of their suitability for each stage of the process.
% section background (end)

\section{Procedure} % (fold)
\label{sec:procedure}
Before you write a Genre Memo, collect several samples of each genre from multiple sources (which may include examples provided by your instructor). Review the samples you collect, looking for trends or patterns in\footnote{from Devitt, Reiff, and Bawarshi's \emph{Scenes of Writing: Strategies for Composing with Genres} (2004), p. 65}:
\begin{itemize}
	\item typical content (included vs. omitted, avoided, or assumed)
	\item appeals to the audience (using logic, emotion, or credibility)
	\item text structure and organization
	\item document format and presentation
	\item authors' choice of words and sentence style
\end{itemize}

Then, consider why the features you identified are appropriate for the given assignment. Why does that genre work so well for the needs of the course, the assignment, or the research process?

To compose your memo, save trouble by relying on the formatting suggested by a template in your software. Organize the information gathered above into a succinct presentation of your analysis of the genre and its functions within the class or the research process. Provide, on a single page, your understanding of the rationale behind using the genre in question.
% section process (end)


\newpage

\section{Purpose} % (fold)
\label{sec:purpose}
Because you need to analyze the use of many genres throughout the semester, you would benefit from a succinct and predictable format for reporting your analyses. You also need something simple, as the analysis of a genre is a reflection on each major assignment, not the assignment itself. Therefore, we will use the ``Genre Memo'' as a short, semi-formal tool for reporting on each genre used during the semester. These documents will only be seen by your instructor, so they are essentially ``intra-office''. Your instructor can quickly see that you understand why we use each genre for specific purposes throughout the semester without requiring lengthy papers. Memos also require you to organize and simplify your thinking for clarity.
% section purpose (end)


\begin{table}[b]
\small
	\caption{The ``Genre Memo'' Assessment Rubric}\label{tab:rubric}
\begin{tabulary}{\columnwidth}{rLL}
	\toprule  & \textbf{\textsc{Accuracy}} & \textbf{\textsc{Application}} \\


\midrule \textbf{Excellent} & Shows holistic understanding of genre's characteristics & Clearly articulated understanding of genre's purposeful function within larger context of research or classwork	\\
\midrule \textbf{Good} & Includes exhaustive review of genre's characteristics & Evident understanding of genre's functions within larger context of research or classwork 	\\
\midrule \textbf{Adequate} & Provides basic list of genre's characteristics & Connects genre's role with specific needs of assignment only	\\
%\midrule \textbf{D} & Includes superficial or incomplete list of characteristics & Understanding of genre's function limited to chronological or mandated considerations	\\
%\midrule \textbf{F} & Improperly identifies genre characteristics or makes no attempt & Fails to correctly assess role played by the genre	\\


	\bottomrule\\
\end{tabulary}
\end{table}

\end{document}


\section{Assessment} % (fold)
\label{sec:evaluation}
Your genre memos will be assessed according to these standards:
\begin{itemize}
	\item \textbf{Accuracy:} identify characteristics of the genre by researching samples and documentation about style and formatting
	\item \textbf{Application:} apply those characteristics to the genre's specific purpose within the relevant stakeholder group
\end{itemize}
More specific evaluation guidelines can be found in Table~\ref{tab:rubric} below.

% section evaluation (end)

\section{Formatting} % (fold)
\label{sec:formatting}
This assignment is obviously not a traditional paper. Therefore, you should avoid the normal paper-formatting approach. Consider using your software's built-in templates for memos. Whatever solution you use, be sure to include the following traditional memo elements:
\begin{itemize}
	\item \textbf{single}-spaced lines
	\item professional typeface
	\item clear section headings
	\item lists, where appropriate
	\item routing information above a divider
\end{itemize}
% section formatting (end)

\end{document}
