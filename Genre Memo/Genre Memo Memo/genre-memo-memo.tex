\documentclass[10pt, oneside]{texMemo}	%defines this as a memo
\usepackage{chrisfriend-memo} %provides formatting declarations for page, figures, textcolor, comments, etc.
\usepackage{chrisfriend-OTF-support} %provides support for OTF system fonts; incompatible with latex, rtf2latex, and ht4latex

\usepackage[toc]{appendix}
\appendixtitleoff
\usepackage[stable, hang]{footmisc} % allows footnotes in section titles (used in appendices here)
\setlength{\footnotemargin}{0em} % ensures that \thanks is flush left and that footnotes don't look silly.

\usepackage{draftwatermark}

\usepackage{tabulary} % allows for the tables I make rubrics with
\usepackage{booktabs} % allows fancy lines in tables

\usepackage{multicol}

\usepackage{mdwlist}

\usepackage[nodate]{datetime} % allows the \currenttime command; nodate tells it not to mess up date settings
%\lhead{\scriptsize\textsc{\textbf{\textcolor{red}{DRAFT}} \number\year .\number\month .\number\day |\currenttime}}




\memofrom{Chris Friend, Instructor}
%\memodate{\number\day\space\monthname\space\number\year}
\memodate{19 Aug 2013}

\memoto{Students in \textsc{enc 1102}}
\memosubject{Genre Memos}

\begin{document}
%\SetWatermarkText{Sample}

\maketitle
\thispagestyle{empty}
\noindent This memo exemplifies the Genre Memo assignment by reviewing the characteristics of memoranda and their use in this course. Use this example as a general guide when creating your genre memo for each major assignment of the semester.

\subsection*{Purpose} % (fold)
\label{sec:background}
Internal office correspondence, when written, often appears as memoranda, or memos. Memos predate email and serve many of the same purposes (quick and simple communication), but they preserve more of an official feel because they are printed. In this class, \textbf{genre memos} document your understanding of the various genres used in the course and provide an official yet direct method for communicating your thinking about genre.
% section background (end)
\subsection*{Layout/Organization} % (fold)
\label{sec:purpose}
Memos emphasize brevity. Formal memos, if used as the standard communication mode within an office, may be several pages long, but most are only a single page. Memos usually \textbf{contain}:
\begin{multicols}{2}
	\begin{itemize*}
		\item Familiar topics
		\item Professional or official tone (but no stuffiness)
		\item Short sentences and ¶s; brevity
		\item Opening ¶ stating purpose
		\item Lists for simplicity and visual emphasis
		\item Section headings, if warranted
		\item Single line spacing
		\item Blank lines between ¶s
		\item Fields at the top for routing and priority
		\item Author's handwritten initials beside name
	\end{itemize*}

	
%This text taken from Coombs, Whitley, Moore, Fontaine, and Toro
	Layout:

The layout of a research proposal typically includes:


Purpose: What exactly defines and explains the plan of action for our research and how we’re going to prove it to our audience

Writing Conventions: Which explains what to include or exclude in our proposals; whether its personal testimony, facts, scientific or scholarly research, etc. This also includes the type of verbiage to use as well as the complexity of the sentence structure

Layout: This explains how the research proposal is to be formatted in a way that can be understood by anyone who needs to look at it

Application to the Course: This section essentially explains the practicality of our research with which accomplishes our goal in our research by proving our points; ultimately persuading our audience of something

\end{multicols}

\noindent Memos typically \textbf{omit}:
\begin{multicols}{2}
	\begin{itemize*}
		\item Salutations, greetings, closers, or signatures
		\item First-person pronouns (this may not apply)
		\item ¶ indents
		\item The verb ``to be''
		\item The passive voice
		\item Color/graphics (unless in attachments)
	\end{itemize*}
\end{multicols}
% section purpose (end)

\subsection*{Application to the Course} % (fold)
\label{sec:procedure}
As stated on the assignment sheet, you need a succinct and predictable format for reporting repetitive analyses. You also need something simple, so you could focus your efforts on each assignment, rather than on this review of them. Therefore, a short, semi-formal tool, such as a memorandum, is ideal for reporting on each genre used during the semester. Memos also require succinct organization and simplicity of presentation, making assessment simpler.
% section procedure (end)

\end{document} 
