\documentclass[11pt,oneside]{amsart}	%defines this as an article
\usepackage{chrisfriend-comp} %provides formatting declarations for page, headers, figures, textcolor, comments, and bibliographic styles
\usepackage{chrisfriend-OTF-support} %provides support for OTF system fonts; incompatible with latex, rtf2latex, & ht4latex
%\usepackage[utf8]{inputenc} %support for smallamp?

%\usepackage{draftwatermark}

%\usepackage{tabularx}
\usepackage{tabulary} % allows for the tables I make rubrics with
%\usepackage{supertabular}
\usepackage{xtab} % allows tables to span pages
\usepackage{booktabs} % allows fancy lines in tables
\usepackage{rotating} % allows landscape tables
\usepackage{lscape} % allows rotated longtables
\usepackage{multirow} % allows rowspanning
\usepackage{enumitem} % helps with the overview
%\usepackage{paralist}

\title[Final Projects]{Assignment Sheet: Final Projects}
\chead{\scriptsize{\MakeUppercase{Final Projects}}}

\begin{document}
%\bibliographystyle{abbrv}
\thispagestyle{empty}

\vspace{-2in}
%\twocolumn[
\begin{center}
\huge
{\includegraphics[height=1.5\baselineskip]{pegasus.pdf}}

\textbf{Assignment Sheet: Final Projects \& Portfolio}

{\normalsize Chris Friend • \textsc{enc1102} • Fall 2013}
\end{center}
\vspace{1.5\baselineskip}
%] %Use for column-spanning the title

\section{Purpose \& Goals} % (fold)
\label{sec:purpose}
Your final project will wrap up the semester by helping you demonstrate your accomplishments for the class in several formats. Your first task is to revise your research proposal. The semester began with a research proposal that explained what you wanted to learn more about. Now that you have been studying that topic for several weeks, you know far more than you did when you initially created the proposal. It's time to put that new knowledge to good use and create an informed research proposal that more confidently argues your position and suggests the kind of research that would be appropriate and successful to follow the line of inquiry you began in this course. Revise or re-create your original research proposal to include a better sense of the research that already exist on the subject and insights you gathered from your own primary research. Did you learn that different primary research needs to be done? Should the conversation you found in your primary research go in a different direction? Knowing what you know now, what is the next project that needs to be done to continue the research on your issue? That's what you propose in this version of your proposal.

To accompany your research proposal and present your ideas in a different and more easily digestible format for a drastically different audience, you will also create a ``pitch'' for your research plan. For this pitch, you will be using the \href{http://www.pechakucha.org}{Pecha Kucha} format, which restricts presentations to fifteen images\footnote{The Pecha Kucha format specifies twenty slides, but we don't have enough time for that during our final exam period.} (notably \emph{without} bulleted lists) displayed for twenty seconds each. Your job is to condense the necessity and importance of your research into \textbf{exactly} five minutes. Your mini presentation will be presented live in class on the exam date, and you will post a link to your talk in the \href{https://webcourses2c.instructure.com/courses/985581/discussion_topics/1918132}{Research Pitches discussion board} so you can see what your peers have created, get feedback from others, and make your slides accessible for presentation.

That leaves us with the final portfolio and course audit. You've been focusing on the course outcomes after each assignment; now your goal is to package the work you've done in a way that showcases your accomplishments and highlights how you've met the course outcomes. Your portfolio will include evidence of the course outcomes to show that you've learned and done what you were supposed to learn and do in this course. Because your work so far has been done online, your portfolio will be, too. You should consider using Google Docs to simplify the process, but you can use other options, as well. Just be sure the content is open-access and linked together. After revising each component of your portfolio, you'll create a Course Audit that explicitly states how and when you achieved the course outcomes, highlighting the parts of the portfolio that show evidence of your achievement. The Course Audit has its own assignment sheet, so be sure you review it before you start writing.
% section background (end)

\begin{table}[b]%\small
	\caption{Goals of Final Portfolio Components}\label{tab:goals}
\begin{tabulary}{\textwidth}{lLLL}
	\toprule  \textbf{\textsc{Component}} & \mbox{\textbf{\textsc{Audience}}} & \textbf{\textsc{Goals}} & \textbf{\textsc{App or Site to Use}}\\
\midrule	\textbf{Proposal} & Future teacher or funding source & Convince people who support research that your project is well thought-out and ready to be approved/funded. & Word processor or \href{http://docs.google.com}{Google Docs} \\
\midrule	\textbf{Pitch} & Your classmates & Explain your issue, justify additional research, and show that your question is worth answering. & Presentation software or Google Docs + \href{http://www.pechakucha.org/}{Pecha Kucha} \\
\midrule	\textbf{Course Audit} & Friend \& DWR & Explain that you know what was expected of you and how you've met the expectations. & Word processor or \href{http://docs.google.com}{Google Docs} \\
\midrule	\textbf{Portfolio} & Friend \& DWR  & Provide and highlight evidence that you've achieved the course outcomes. & Word processor, \href{http://docs.google.com}{Google Docs}, \href{https://github.com/}{GitHub}, blog, etc. \\
%\midrule	\textbf{Component} & Audience & Goals & Apps \\
	\bottomrule
\end{tabulary}
\end{table}


\section{Procedure} % (fold)
	\label{sec:procedure}
This assignment sheet covers several final activities, so be sure you understand each component before you start working. You may decide that the order presented below won't work for you. That said, I recommend completing tasks in this order:
\begin{description}
	\item[Write Research Paper] Draft your Academic Research Report to include all findings from this semester. Your work will be included in the class journal that Friend will produce and distribute at the end of the semester. As you write, emulate the style, expectations, and other conventions of academic writing.
	\item[Update Proposal] Next, re-create the proposal you wrote at the beginning of the semester. Considering you now know more than you did back then, create a proposal for the whatever research should be conducted next with regard to your topic. If necessary, schedule a conference to discuss your revision plans or progress with your instructor.
	\item[Build Pitch] Now, shift gears. After explaining why your research needs to be done using academic terms and an academic genre, now you need to do the same thing that for a different audience and with a different genre. Given exactly five minutes and exactly twenty slides, how can you explain the work you did this semester and justify the need for your proposed research project using only images and your voice, rather than text? Create a \href{http://www.pechakucha.org/}{Pecha Kucha}-format presentation using presentation software of your choice. Set the slides to advance automatically every 15 seconds, then post a link on the \href{https://webcourses2c.instructure.com/courses/985581/discussion_topics/1918132}{Research Pitches discussion board} so your classmates can see what you've created.
	\item[Write Course Audit] Now that you have collected all of your documents from the semester, you need to write a cover letter that articulates how those documents demonstrate that you have achieved the intended outcomes of the course. Write a letter to your instructor that helps guide him through a review of your portfolio, pointing out where you achieved each outcome listed on syllabus. Remember: the goal here is to demonstrate that you achieve the outcomes. Do not write a letter that emphasizes flattery or informs the instructor that this course has been your most breathtakingly life-changing experience to date, that you don't know what you would do without the amazing assistance or instructor gave the semester, etc. Although any of these situations may be true, pointing them out sounds more like kissing up and less like completing the assignment. Resist the urge.
	\item[Assemble Portfolio]  This is actually the easy part. Now that your proposal is finalized, all you need to do is compile all of the documents you've created this semester into one manageable package. Your Course Audit serves as a cover sheet and should go first. Create a table of contents afterward, and make sure the arrangement of your documents reflects your view of the course as expressed in your Course Audit. How you present the whole package is up to you, but I recommend sticking with what is simple and familiar: Google Docs works great.
	\begin{itemize}
		\item Using Google Docs?
		\begin{enumerate}
			\item Open Friend's \href{https://docs.google.com/document/d/1iAnB0vb4Jo_ezOEuCqSu1nGX4AfqvYvmD1BSNna1SzM/edit?usp=sharing}{portfolio template}.
			\item Choose ``Make a copy…'' from the File menu; save your portfolio to your own Google account.
			\item Below the placeholder for your Course Audit, add links to your other documents, and link to or embed your Research Pitch.
			\item Write your Course Audit. (Details on separate assignment sheet.)
			\item \textbf{Make absolutely certain Sharing settings allow Anyone with the Link to Edit.}
		\end{enumerate}
		\item Have another idea for presenting your portfolio, like \href{https://github.com}{GitHub}, \href{https://sites.google.com}{Google Sites}, etc.? Check with your instructor. The format of this document is very flexible, so your ideas will probably work fine.
	\end{itemize}
	\item[Publish \& Submit Everything] Inside Webcourses, submit your work to the \href{https://webcourses2c.instructure.com/courses/985581/assignments/2821738}{Final Projects assignment} so that your instructor knows it is ready to be graded. For any work that is published elsewhere on the web (such as on YouTube or Google Docs), all you need do is submit a link that your instructor can follow. For documents that do not already live online (such as a portfolio created in a word processor), you will need to upload the document into Webcourses. \textbf{For any work published online, make absolutely certain access is open to anyone with a link.}
	\item[Relax] Enjoy the feeling of being finished with the course!
\end{description}
% section procedure (end)

\section{Evaluation} % (fold)
\label{sec:evaluation}
Your performance on these final projects determines your score on the ``products'' portion of your final grade. Specific details for grading criteria appear in Table~\ref{tab:rubric}, but be sure to review all feedback you received from your instructor and your peers throughout the semester to ensure your final draft is as effective as possible.
% section rubric (end)

\begin{table}[b]%\small
	\caption{Evaluation of Final Products}\label{tab:rubric}
\begin{tabulary}{\textwidth}{rL}
	\toprule  & \textbf{\textsc{Demonstration of Course Outcomes}} \\
\midrule	\textbf{A} & Clearly demonstrates exceptional achievement of course outcomes and provides easily identifiable evidence of those outcomes across all submitted materials. \\
\midrule	\textbf{B} & Demonstrates strong mastery of course outcomes, made obvious through the content of all submitted materials. \\
\midrule	\textbf{C} & Shows attainment of course outcomes and provides documentation of those accomplishments through satisfactory attached documents. \\
\midrule	\textbf{D} & Provides evidence of an unsuccessful attempt to achieve the course outcomes. Supporting documents likely haphazard, of poor quality, or incomplete. \\
\midrule	\textbf{F} & Fails to demonstrate achievement of course outcomes or to submit a complete portfolio. Work is of unacceptably bad quality. \\
	\bottomrule
\end{tabulary}
\end{table}
% section rubric (end)

\end{document}